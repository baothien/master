\section{Introduction}
\label{sec:introduction}
The amount of visual information available in digital collections
increases every day. Pictures, medical images, graphs and photos
among others are collected for books, papers, diagnosis, reports,
news, albums, etc. Due to this, Content-Based Image Retrieval
(CBIR) has recently become an active research discipline. Despite
the advances in content-based image retrieval, different problems
remain unsolved. The main problem is the semantic interpretation of
the image content~\cite{smeulders2000content}. Most of CBIR systems try to work on
semantic of image. But the semantic is not easy to get. There are
many solutions for understanding the content of image. The easiest
way is to describe image with text annotation. There are different
situations in which images are surrounded by text descriptions that
may provide contextual hints or semantic information about their
contents. In this approach, image semantics are learned from an
external source such as user feedbacks and textual annotations, and
of course both are provided by human beings. 
However, learning from external sources is limitted because we have no
annotation is correct or not. Moreover, there are different scenarios
in which descriptive annotations are not available for all images in
the collection or the user is not able to provide an accurate textual
query. Furthermore, several works have shown that, even in the
presence of text descriptions, visual features are useful
information source to identify relevant images in a large scale CBIR
system, showing important improvements in performance~\cite{jing2007canoical,grubinger2008advances}.

In this research, to understand the semantic of image, instead of
annotation, we base on objects in the image. The more objects a
system can detect, the more precise that system is. Object detection
and recognition is still a difficult problem for computer vision. Most
methods base on features of an image including both local and
global one for detecting object. The goal of this research is to
develop a necessary methodology for recognition of general object
in still images by automatically choosing features. We base on
features that are easily with natural recognizing by human. Most
state-of-art methods focus on how to recognize object after fixing
some features. In contrast, in this research, we let system choose
features from a list of features, and the system determines by itself
which features are good for detecting a given object.

The rest of this paper is organized as follows. Section~\ref{sec:related_works} discusses
previous work related to the approach taken here. Section 3
describes the set of attributes we use, and how we create attribute
classifiers. In section 4 we propose an algorithm for automatically
choosing best features. In our experiments (section 5) we first
evaluate the performance of both the individual attribute classifiers
and combination attribute classifiers. We also compare the attribute features 
with other methods. Result of automatically choosing features is also presented in this part.
The last section discusses experiment results and suggests some future directions.
